\documentclass[12pt]{amsart}
\usepackage{geometry}% [paperheight=5in]     % See geometry.pdf to learn the layout options. There are lots.
%\geometry{letterpaper}                   % ... or a4paper or a5paper or ... 
%\geometry{landscape}                % Activate for for rotated page geometry
%\usepackage[parfill]{parskip}    % Activate to begin paragraphs with an empty line rather than an indent
\usepackage{graphicx}
%\usepackage[parfill]{parskip}
\usepackage{amssymb}
\usepackage{tikz}
\usepackage{cite}
\usepackage{algpseudocode}

\newtheorem{theorem}{Theorem}[section]
\newtheorem{lemma}[theorem]{Lemma}
\newtheorem{proposition}[theorem]{Proposition}
\newtheorem{corollary}[theorem]{Corollary}

\newenvironment{definition}[1][Definition]{\begin{trivlist}
\item[\hskip \labelsep {\bfseries #1}]}{\end{trivlist}}
\newenvironment{example}[1][Example]{\begin{trivlist}
\item[\hskip \labelsep {\bfseries #1}]}{\end{trivlist}}
\newenvironment{remark}[1][Remark]{\begin{trivlist}
\item[\hskip \labelsep {\bfseries #1}]}{\end{trivlist}}

\begin{document}
\begin{definition}
Let $\mathbb{K}$ denote the field of Puiseux series in $t$ with complex coefficients. A monomial map $\phi:\mathbb{K}^n\rightarrow \mathbb{K}^m$, $(x_1,\ldots,x_n)\mapsto (\mathbf{x^{a_1}},\ldots,\mathbf{x^{a_m}})$ where $a_i\in\mathbb{Z}^n$ is a \textit{good projection} of $P\subseteq\mathbb{K}^n$ if $m=n-1$, im $\phi$ has dimension $n-2$ as a toric map, and $\phi$ is injective on the set of leading terms of elements of $P$.

------------------------or------------------------

A monomial map $\phi:T^n\rightarrow T^{m}$ is a \textit{good projection} of a set of initial Puiseux tuples $P\subseteq\{(a_1t^{m_1},\ldots,a_nt^{m_n}):a_i\in\mathbb{C^*},m_i\in\mathbb{Z}\}$ if $m=n-1$, im $\phi$ has dimension $n-2$ as a toric map, and $\phi$ is injective on $P$.
\end{definition}
\begin{proposition}
Good projections exist when $P$ is finite.
\end{proposition}

\begin{proof}
To prove the proposition, we will characterize those monomial maps which fail to be injective on S. We begin by considering $\phi$ as a map $\mathbb{C}^n\rightarrow\mathbb{C}^{n-1}$ and examine its effect on the coefficients of elements of $P$. First note that if $\phi: (x_1,\ldots,x_n)\mapsto (p_1(\mathbf{x}),\ldots,p_m(\mathbf{x}))$ where the $p_i$'s are Laurent monomials, then $\phi$ is injective on the set $C$ of coefficient tuples if and only if $p_i(a)\neq1$ for all $a$ in $S=\{(\frac{a_1}{b_1},\ldots,\frac{a_n}{b_n}):a,b\in C\}$. This follows from the fact that monomial maps commute with multiplication. Thus $\phi$ is injective on coefficients if and only if no element of $S$ lies in the preimage of $\mathbf{1}=(1,\ldots,1)$.

Each point of $S$ is nonzero in all coordinates, thus lies in the torus and thus is in the preimage of $\mathbf{1}$ if and only if it is equal to $(t^{k_1},\ldots,t^{k_n})$ for some $t\in\mathbb{C}$, where $\mathbf{k}$ generates the 1-dimensional kernel of the $n\times n-1$ matrix of the exponents of the monomials of $\phi$.

Each $\mathbf{c}\in S$ therefore gives a system of exponential equations $t^{k_i}=c_i$. Choosing a branch of the logarithm that misses all of the $c_i$, we find that $\log t=\frac{\log c_i}{k_i}$ for all $i$. Simple manipulation yeilds $c_i=c_1^{k_i/k_1}$ for all $i$. Define $d:=c_1^{1/k_1}$. If $\mathbf{c}=(s^{j_1},\ldots,s^{j_n})$ for $j_i\in\mathbb{Z}$, then $t^{k_i}=d^{k_i}=s^{j_i}$ and thus $s$ is a $\mathbb{C}$-multiple of $t$ and $j_i$ is a $\mathbb{Q}$-multiple of $k_i$. So at most one choice of $\phi$, up to scaling of the exponents, will fail to be injective on $S$.

We now consider $\phi$ as a map $\mathbb{Z}^n\rightarrow\mathbb{Z}^{n-1}$ of exponent rays. Choosing a ray $\mathbf{r}$ in $\mathbb{Z}^n$, we can form a $\mathbb{Q}$-linear map by projecting into the hyperplane perpendicular to $\mathbf{r}$, taking the unimodular coordinate transformation to bring that plane parallel to the first coordinate hyperplane, and dropping the first coordinate. As a composition of linear maps this is clearly linear; by scaling we can ensure that all entries in its defining matrix are in $\mathbb{Z}$.

Let $R$ be the set in $\mathbb{Q}^n$ of rays generated by the exponent tuples of all possible pairs of elements of $P$. If we choose $\mathbf{r}$ from in $\mathbb{Z}^n\setminus R$, the map will be injective on the exponents of elements of $S$ by construction. Since at most one map fails to be injective on coefficients, we have shown that many good projections exist.
\end{proof}



%$\mathbb{A}\mathbb{B}\mathbb{C}\mathbb{D}\mathbb{E}\mathbb{F}\mathbb{G}\mathbb{H}\mathbb{I}\mathbb{J}\mathbb{K}\mathbb{L}\mathbb{M}\mathbb{N}\mathbb{O}\mathbb{P}\mathbb{Q}\mathbb{R}\mathbb{S}\mathbb{T}\mathbb{U}\mathbb{V}\mathbb{W}\mathbb{X}\mathbb{Y}\mathbb{Z}$

\end{document}
